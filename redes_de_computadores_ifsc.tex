\documentclass{article}
\usepackage[portuguese]{babel}
\usepackage[letterpaper,top=2cm,bottom=1cm,left=3cm,right=3cm,marginparwidth=1.75cm]{geometry}

\usepackage{amsmath}
\usepackage{graphicx}
\usepackage[colorlinks=true, allcolors=blue]{hyperref}

\title{Sobre Redes de Computadores}

\begin{document}
\maketitle

\section{Conceitos Importantíssimos}
\begin{itemize}
    \item Em redes, as unidades de medida mais usadas são o bit e o bit/s. Byte não é muito usado
    \item É importante saber a história das redes e sistemas. Mesmo que os sistemas, sejam hardwares ou softwares, não sejam mais usados. Porque eles que fazem a gente entender o presente
\end{itemize}

\section{História das Redes de Computadores}
As primeiras redes de computadores surgiram nos Estados, durante a guerra fria, antes mesmo da popularização dos computadores! As redes surgiram devido a necessidade de comunicação em caso de uma guerra nuclear. As universidades americanas e os militares se comunicavam. Após o fim insistência das universidades

\section{LAN x MAN x WAN}
Inicialmente só existia LAN e WAN, aí meio acidentalmente foi criado MAN
\begin{itemize}


\item LAN (Local Area Network):
- Rede local. Distâncias curtas. Em redes de computadores, uma distância curta é uma distância < 2km. O tipo mais comum é o Ethernet
- Primeiro sistema LAN: Ethernet
- Sucessor: Fibra óptica. Ela aumentou tanto o alcance da rede, que criaram o tipo "MAN". Mas inicialmente, era pra fibra óptica ser uma rede LAN. A fibra óptica usa um laser de alta potência para o envio do sinal

\item WAN (Wide Area Network):
- Rede de longa distância. Abrange redes no intervalo de [2km, $\infty$[. Sim, redes maiores que 2km, ATÉ O INFINITO, são redes WAN

\item MAN (Metropolitan Area Network):
- Foi uma classificação criada posteriormente, já que 2km até o infinito é um intervalo muito grande pra uma classificação só. Redes MAN um meio termo entre LAN e WAN, em suma. Apesar disso, o termo MAN é pouco utilizado
\end{itemize}

\section{Redes Privadas x Redes Públicas}
Meio óbvio

\section{Repeaters, Bridges e Learning Bridges}

\subsection{Repeaters (repetidores)}
Atuam na camada 1 do ISO/OSI. Eles amplificam o sinal de uma rede. ELe não cria informações, ele não filtra nada,e el só amplifica, é como se ele fosse \textbf{um mega fone} que é colocado estrategicamente onde o sinal da rede estiver fraco

\subsection{Bridges}
Bridges atuam na camada 2. Eles não amplificam sinal de uma rede, diferentemente de repetidores. Eles conectam duas redes locais (LANs) ou dois pedaços de uma mesma rede LAN. Eles servem para organizar as redes

\subsection{Learning Bridges}

\section{Switches}
Switches são

\section{O Conceito de MODEM}
O Modem é um aparelho eletrônico que modula e demodula frequências, daí que vem o nome MO-DEM, uma sigla para modeling, demodeling

\section{SLIP (Serial Line Internet Protocol) e PPP (Point-to-Point Protocol)}

\subsection{SLIP (Serial Line Protocol)}
\textbf{O SLIP surge para solucionar um problema:} Criar redes é fácil, mas WAN é muito difícil (longas distâncias), custo alto e baixa velocidade
\\[10pt]
\textbf{A ideia do SLIP:} Utilizar as linhas telefônicas já existentes para transmitir dados da internet por uma linha serial RS232 

\subsection{PPP (Point-to-Point Protocol)}
É um protocolo de camada 2, ele pode utilizar o mesmo meio físico do ethernet, inclusive é chamado de "PPP sobre ethernet"
\\[10pt]
É mais avançado que o SLIP, sendo capaz de encapsular diversos protocolos (não apenas IP)

\section{IP (Internet Protocol)}
O IP é um protocolo
\\[10pt]
\textbf{O Internet Protocol (IP) é como um sistema de endereços de entrega:} É como o sistema de correios. Ele é responsável por dizer onde o pacote deve chegar. Para isso, o IP dá um endereço único a cada computador e define as regras para o roteamento dos pacotes entre redes
\\[10pt]
Ele define como os dados são endereçados e roteados \textbf{(rotear = escolher o caminho que os dados vão seguir para chegar ao destino)} pela rede
\\[10pt]
O IP dá um endereço único a cada dispositivo \textbf{(o famoso endereço IP)}, e garante que os pacotes de dados sejam enviados de um dispositivo de origem para o destino correto, mesmo passando por várias redes
\\[10pt]
\textbf{ATENÇÃO:} Popularmente, nos referimos ao endereço de IP simplesmente como "IP". Falamos tipo: "Ah o meu IP é 000.000.00". Isso está incorreto. Não confunda o \textbf{endereço de IP} com o \textbf{IP (Internet Protocol)}, um é apenas um endereço utilizado pelo protocolo, e o outro é o protocolo em si
\\[10pt]
Algumas características do IP são:

\begin{itemize}
    \item Sem conexão (não há rota pré-estabelecida)

    \item Não confiável - a entrega não é garantida

    \item Máximo esforço - apesar de não ser garantido, os sistemas na rede irão fazer o máximo de esforço para entregar o pacote. Ele nunca será jogado fora propositalmente, somente se houver algum problema no software
\end{itemize}

\section{DNS (Domain Name System)}
O DNS (Domain Name System) é como uma ``agenda telefônica" da Internet. Ele converte nomes de domínio (como www.google.com) em endereços IP (como 142.250.217.78), que então são usados pelos computadores para identificar e se conectar a servidores na rede.
\\[10pt]
\textbf{Como o DNS funciona:}
\begin{enumerate}
    \item Consulta DNS: Quando você digita um endereço DNS no navegador, uma consulta é enviada para descobrir o endereço IP correspondente ao nome de domínio
    
    \item Hierarquia DNS: O DNS usa uma estrutura hierárquica:
    \begin{itemize}
        \item Root Servers: São os servidores principais que redirecionam as consultas para o domínio de topo (TLD), como .com, .org, .br
        \item TLD Servers: Eles encaminham para o servidor DNS que gerencia o domínio específico.
        \item Servidores Autoritativos: São os servidores finais que retornam o endereço IP do domínio solicitado.
    \end{itemize}
    
    \item Cache: Para acelerar o processo, sistemas locais e servidores armazenam endereços IP de consultas recentes
\end{enumerate}
Quando você digita um endereço como \texttt{http://x.y.z}, o DNS resolve o nome \textbf{de trás pra frente}, começando da parte mais \textbf{genérica} para a mais \textbf{específica}:

\begin{enumerate}
    \item \textbf{Parte mais genérica}:  
    Ele olha primeiro para o \textbf{final} do endereço, o \texttt{.z} (isso pode ser \texttt{.com}, \texttt{.br}, etc.). Essa parte é chamada de \textit{TLD (Top-Level Domain)}.
    
    \item \textbf{Parte intermediária}:  
    Depois, ele vai para a parte antes do TLD, que é o \texttt{y}. Aqui ele pergunta: “Quem é responsável por gerenciar o domínio \texttt{y.z}?”
    
    \item \textbf{Parte específica}:  
    Por fim, chega no \texttt{x}, que é o domínio mais específico. Aqui, ele descobre exatamente qual servidor está associado a \texttt{x.y.z}. Por exemplo \texttt{.google}
\end{enumerate}
Depois de encontrar o endereço IP correspondente, ele devolve isso para o navegador, que usa o IP para se conectar ao servidor e carregar o site.

\end{document}
